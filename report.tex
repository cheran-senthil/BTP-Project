\documentclass[12pt,a4paper]{report}

\usepackage{amsthm,amssymb,mathrsfs,setspace,amsmath,mathtools}%amsmath, latexsym,footmisc
\DeclarePairedDelimiter{\ceil}{\lceil}{\rceil}
% \usepackage{pstcol}
% \usepackage{play}
\usepackage{epsfig}
%\usepackage[grey,times]{quotchap}
\usepackage[nottoc]{tocbibind}
\renewcommand{\chaptermark}[1]{\markboth{#1}{}}
\renewcommand{\sectionmark}[1]{\markright{\thesection\ #1}}
%

\input xy
\xyoption{all}


\theoremstyle{plain}
\newtheorem{theorem}{Theorem}[section]
\newtheorem{lemma}[theorem]{Lemma}
\newtheorem{corollary}[theorem]{Corollary}
\newtheorem{proposition}[theorem]{Proposition}
\newtheorem{conjecture}[theorem]{\bf{Conjecture}}

\theoremstyle{definition}
\newtheorem{definition}[theorem]{Definition}
\newtheorem{example}[theorem]{Example}
\newtheorem{notation}[theorem]{Notation}

\theoremstyle{remark}
\newtheorem{remark}[theorem]{Remark}

\renewcommand{\baselinestretch}{1.5}

\usepackage{tocloft}

\makeatletter
\@addtoreset{section}{part}
\makeatother
\newlength\mylen
\renewcommand\thepart{\arabic{part}}
\renewcommand\cftpartpresnum{Part~}
\settowidth\mylen{\bfseries\cftpartpresnum\cftpartaftersnum}
\addtolength\cftpartnumwidth{\mylen}


\begin{document}

%\pagenumbering{arabic} \setcounter{page}{1}

% --------------- Title page -----------------------

\begin{titlepage}
\enlargethispage{3cm}

\begin{center}

\vspace*{-2cm}

\textbf{\Large Congruences of Partitions:} \\
\textbf{\Large Rank Differences and Cubic Partition Pairs}

\vfill

A Project Report Submitted \\
in Partial Fulfilment of the Requirements \\
for the Degree of \\[1cm]

{\bf\Large\ BACHELOR OF TECHNOLOGY }\\[.1in]
{\large {in}}\\[5pt]
{\large\bf {Mathematics and Computing}}\\[5pt]

 \vfill

{\large \emph{by}}\\[5pt]
{\large\bf {Abhishek Tyagi and Seralathan V S}}\\[5pt]
{\large (Roll No. 140101004 and Roll No. 140123032)}

\vfill
\includegraphics[height=2.5cm]{iitglogo}

\vspace*{0.5cm}

{\em\large to the}\\[10pt]
{\bf\large DEPARTMENT OF MATHEMATICS} \\[5pt]
{\bf\large \mbox{INDIAN INSTITUTE OF TECHNOLOGY GUWAHATI}}\\[5pt]
{\bf\large GUWAHATI - 781039, INDIA}\\[10pt]
{\it\large April 2018}
\end{center}

\end{titlepage}

\clearpage

% --------------- Certificate page -----------------------
\pagenumbering{roman} \setcounter{page}{2}
\begin{center}
{\Large{\bf{CERTIFICATE}}}
\end{center}
%\thispagestyle{empty}


\noindent
This is to certify that the work contained in this project report entitled ``{\bf Congruences of Partitions: Rank Differences and Cubic Partition Pairs}" submitted by {\bf Abhishek Tyagi} ({\bf Roll No. 140101004}) and {\bf Seralathan V S} ({\bf Roll No. 140123032}) to Department of Mathematics, Indian Institute of Technology Guwahati and Department of  towards the towards partial requirement of \textbf{Bachelor of Technology} in Mathematics and Computing has been carried out by them under my supervision.
\\\\
\noindent
It is also certified that this report is a survey work based on the references
in the bibliography.\\\\
\noindent
Turnitin Similarity: 18 \%

\vspace{4cm}

\noindent Guwahati - 781 039 \hfill (Dr. Rupam Barman)

\noindent April 2018 \hfill Project Supervisor

\clearpage

% --------------- Abstract page -----------------------
\begin{center}
{\Large{\bf{ABSTRACT}}}
\end{center}

This report describes the work of our B.Tech Project in the field of Partition Theory specifically in the study of Congruences of Partition Rank differences and Cubic Partition Pairs.

The celebrated Ramanujan's congruences led Dyson to conjecture the existence of the rank of a partition such that we get a combinatorial interpretation to these congruences in [10]. The notion of rank was extended to rank differences and $M_2$ rank differences upon which Mao conjectured several inequalities in [18, 19].

We investigated the proof to one of Mao's ten conjectures in [19] and attempted to extend the solution given by Barman and Sachdeva in [3] by using the properties of $\varphi^2(q)$, in which the coefficient of $q^n$ counts the number of Integer solutions to $n = a^2 + b^2$ or using general proof for $M_2$-rank differences given in [1].

We also investigated the conjectures made by Lin in [17] before finding a proof for the same in [11].

\clearpage



\tableofcontents
\clearpage


\newpage

\pagenumbering{arabic}
\setcounter{page}{1}

% =========== Main chapters starts here. Type in separate files and include the filename here. ==
% ============================

\part{}

\input chapter1.tex

\input chapter2.tex

\input chapter3.tex
\addtocontents{toc}{\protect\newpage}
\input chapter4.tex

\part{}

\input chapter5.tex

\input chapter6.tex

\nocite{atkin1} \nocite{atkin2} \nocite{barman} \nocite{berndt} \nocite{dyson}  \nocite{mao1} \nocite{mao2} \nocite{Lin2017} \nocite{chan1} \nocite{chan2} \nocite{chan3} \nocite{kim1} \nocite{kim2} \nocite{kim3} \nocite{zhao1} \nocite{gireesh} \nocite{cooper} \nocite{hwan} \nocite{sellers} \nocite{berndt2}

\bibliographystyle{plain}
\bibliography{refer_items}

\end{document}

