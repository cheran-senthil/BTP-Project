\chapter{Proof of Lin's Conjectures}

\section{Lin's Conjectures}

In [17] Lin conjectured the following:
\begin{conjecture}
\begin{equation}
    b(81n + 61) \equiv 0 \pmod{243}.
\end{equation}
\end{conjecture}

\begin{conjecture}
\begin{equation}
    \sum_{n=0}^\infty b(81n + 7)q^n \equiv 9 \frac{(q^2;q^2)_\infty(q^3;q^3)^2_\infty}{(q^6;q^6)_\infty} \pmod{81},
\end{equation}
\begin{equation}
    \sum_{n=0}^\infty b(81n + 34)q^n \equiv 36 \frac{(q;q)_\infty(q^6;q^6)^2_\infty}{(q^3;q^3)_\infty} \pmod{81}.
\end{equation}
\end{conjecture}

\newpage

\section{Gireesh and Naika's proof}
In [11], Gireesh and Naika present elegant proof for general family of congruences modulo large powers of 3 for cubic partition pairs.

They use results obtained by Cooper in [9] and the "huffing" operator H to arrive at the following general congruences.
\begin{theorem}
For each $\alpha \geq 0$ and $n \geq 0$
\begin{equation}
    b\bigg(3^{2\alpha + 1}n + \frac{3^{2\alpha + 1} + 1}{4}\bigg) \equiv 0 \pmod{3^{2\alpha}}
\end{equation}
\begin{equation}
    b\bigg(3^{2\alpha + 4}n + \frac{3^{2\alpha + 5} + 1}{4}\bigg) \equiv 0 \pmod{3^{2\alpha + 5}}
\end{equation}
\begin{equation}
    b\bigg(3^{2\alpha + 5}n + \frac{7\cdot 3^{2\alpha + 4} + 1}{4}\bigg) \equiv 0 \pmod{3^{\alpha + 6}}
\end{equation}
\begin{equation}
    b\bigg(3^{2\alpha + 5}n + \frac{11\cdot 3^{2\alpha + 4} + 1}{4}\bigg) \equiv 0 \pmod{3^{\alpha + 6}}
\end{equation}
\end{theorem}

Now Lin's Conjecture 6.1.1 follows directly from (6.5) by substituting $\alpha = 0$.

\begin{proof}
From [11], we elaborate on the proof of Lin's conjecture 6.1.2
\begin{lemma}
\begin{equation}
        \frac{f^5_2}{f^4_1} = \frac{f^6_6f^{10}_9}{f^{10}_3f^5_{18}} + 4q\frac{f^5_6f^7_9}{f^9_3f^2_{18}} + 9q^2\frac{f^4_6f^4_9f_{18}}{f^8_3} + 10q^3\frac{f^3_6f_9f^4_{18}}{f^7_3}+4q^4\frac{f^2_6f^7_{18}}{f^6_3f^2_9}
\end{equation}
\begin{proof}
From [i, p. 49, Corollary],
\begin{equation}
    \frac{f^2_2}{f_1} = \frac{f_6f^2_9}{f_3f_{18}} + q\frac{f^2_{18}}{f_9}
\end{equation}
In [12], Hirschhorn proves,
\begin{equation}
    \frac{f_2}{f^2_1} = \frac{f_6^4f_9^6}{f_3^8f_{18}^3} + 2q\frac{f^3_6f_9^3}{f^7_3}+4q^2\frac{f^2_6f^2_{18}}{f^6_3}
\end{equation}
By squaring (6.9), then multiplying by (6.10), we obtain (6.8)
\end{proof}
\end{lemma}

\begin{lemma}
From [13, Eq. 5.1], we have,
\begin{equation}
    \frac{f_2^4f_3^8}{f_1^8f_6^4} + q\frac{f_2f_6^5}{f_1^5f_3} \equiv 1 \pmod 9
\end{equation}
\end{lemma}

\begin{lemma}
\begin{equation}
\begin{split}
    \sum_{n = 0}^\infty b(27n + 7)q^n &\equiv 9\frac{f_3^2f_6^2}{f_1^1f_2^4} 
    \\&\equiv 9 \frac{f_3^2f_2^5}{f_1^4f_6} \pmod{81}
\end{split}
\end{equation}
\end{lemma}

Using (6.8) and (6.12), we get,
\begin{equation}
    \sum_{n=0}^\infty b(27n + 7)q^n \equiv 9 \frac{f_3^2}{f_6}
    \bigg(\frac{f_6^6f_9^{10}}{f_3^{10}f_{18}^5} + 4q\frac{f^5_6f_9^7}{f^9_3f_{18}^2}+q^3\frac{f^3_6f_9f^4_{18}}{f^7_3}+4q^4\frac{f_6^2f_{18}^7}{f_3^6f_9^2}\bigg) \pmod{81}
\end{equation}
Through which we get,
\begin{equation}
\begin{split}
    \sum_{n=0}^\infty b(81n + 7)q^n &\equiv 9 \frac{f_1^2}{f_2}
    \bigg(\frac{f_2^6f_3^{10}}{f_1^{10}f_6^5} + +q\frac{f^3_2f_3f^4_6}{f^7_1}\bigg)
    \\&\equiv 9 \frac{f_2f_3^2}{f_6}
    \bigg(\frac{f_2^4f_3^8}{f_1^8f_6^4} + +q\frac{f_2f^5_6}{f^5_1f_3}\bigg) \pmod{81}
\end{split}
\end{equation}
Using (6.11) in (6.14), we arrive at (6.2)
\newpage
Taking the terms from (6.13) in which the powers of $q$ are congruent to 1 modulo 3, then dividing by $q$, and then finally replacing $q^3$ by $q$, we get,
\begin{equation}
\begin{split}
\sum_{n=0}^\infty b(81n + 34)q^n &\equiv 36 \frac{f_1^2}{f_2}
    \bigg(\frac{f_2^5f_3^7}{f_1^9f_6^2} + +q\frac{f^2_2f_3f^7_6}{f^6_1f_3^2}\bigg)
    \\&\equiv 36 \frac{f_1f_6^2}{f_3}\bigg( \frac{f_2^4f_3^8}{f_1^8f_6^4} + q\frac{f_2f_6^5}{f_1^5f_3} \bigg)
\end{split}
\end{equation}

Using (6.11) in (6.15), we arrive at (6.3).

And thus we complete the demonstration of the proof for Conjecture 6.1.2

\end{proof}